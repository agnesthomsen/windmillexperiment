\documentclass[12pt]{article}

\usepackage{mystyle}

\title{\textbf{IB Internal Assessment}\\\textbf{Physics SL}\\ \vspace{5cm}Measuring the efficiency of wind turbines}
\author{\vspace{5cm}\\Agnes H\o y-Thomsen\\Candidate Number: 000115-0026\\Word Count: }

\pagestyle{fancy}
\fancyhead{}
\fancyfoot{}

\fancyhead[L]{\tiny Agnes H\o y-Thomsen\\Physics SL\\Candidate Number: 000115-0026}
\fancyfoot[C]{\thepage}

\begin{document}

\maketitle
\clearpage
\vspace{5cm}
\tableofcontents
\newpage

%AGNES: START WRITING HERE
\section{Introduction} 

Global warming has been a heavily debated issue the past couple of years with increasing concern for the future. With the poles melting and the sea levels rising, but also the ozone layer is disappearing slowly and the planet is heating up leaving behind species now in danger of extinction, there are clear signs of change. Sea levels is particularly a problem I worry about in my country, Denmark, which is surrounded by sea and has a flat surface and already struggles with sea slowly eating up the coast lines. With fossil fuels being 87 per cent of our primary energy consumption, according to a study in 2012, it’s clear that there is a limited amount of alternatives to compete with them. Besides this argument, CO$_2$ emission not only affects the environment, but also our health; we seem to care more about short-term profits than our and the next generations’ health. Nuclear power is a great alternative producing enormous amounts of energy, but radiation release in terms of emergences is highly dangerous and waste material has to be hidden deep ground or in mountains.

Renewable energy is increasing every year and lots of money is invested in new technology to improve the world’s savers in the future. Not only does renewable energy help reducing the global warming, it also uses sources that are free and everlasting. I was focusing on wind turbines and how efficient they are, which is one of the alternatives to fossil fuels. It is an interesting topic, since it is such a current one and this will hopefully increase my area of knowledge about the wind turbine and its function in our world, and therefore, I will investigate the efficiency of wind turbines.


\section{Background Theory}
How is wind created?
The sun emits radiation, where some of this the surface of the Earth absorbs and the surface heats up. However, due to mountains, desserts and hills the heating up is different and therefore depending on location, the temperature varies. Where the temperature is higher, the air rises and creates low pressure. Places with cooler temperatures have more dense air of molecules and do therefore not rise, creating a high pressure. The movement of air from high to low pressure is also convectional currents and creates wind.

Albert Betz discovered the maximum theoretical output of a wind turbine being 59,3\%, which is known as Betz’s Coefficient.  However, this is a theoretical output. It is estimated more realistically that a wind turbine, under Danish circumstances, only uses and converts an average of 25-30\% of the actual energy available to it throughout the year, yet modern turbines can use up to 45\% of the wind on optimal days.  However, due to around 6-10\% loss of energy in the gearbox and the generator, the realistic average of power is between 22-28\%. 
\begin{equation}
P = \frac{1}{2} \rho A v^3
\label{equation:Windpower}
\end{equation}
This formula is the power of the wind that reaches the turbines within the area of the wings, which is quite useful, since not all wind goes to the turbine, but only that within the area. $P$ is power, $\rho$ is air density, $A$ is the area of the blades with $r$ being the length of one blade, and $v$ is the velocity of the wind. The average wind speed for it to turn is around $6-8 ms^{-1}$ throughout the year. This equation can be seen as the input of the turbine.


\begin{equation}

P=VI
\label{equation:Power}
\end{equation}


This is another formula to calculate power, where $P$ is power, $V$ is voltage and $I$ is current. This formula says that if voltage and current is multiplied it gives power, which leads to:

R=V/I
In a circuit, we can calculate the resistance, where R is resistance, V is voltage and I is current.

Mechanics:
The wind makes the blades turn the main shaft converting kinetic energy into mechanical energy. The gearbox gets the speed up to 3000 rotations per minute, which is 50hz. Then it goes to the generator, which converts mechanical into electric energy.
The blades are bent backwards slightly to achieve that the wind pushes the blades down to make them turn. 
The winds can be pitched regulated in order to control power output. This means that the wings can turn to reach a specific angle between the blade and the wind, the pitch angle. This makes the efficiency of the turbine constant during high wind speeds over 14ms^(-1). Even though it does not drop in efficiency it does, however, mean that even though the wind speed increases with more than 14 ms^(-1), the turbine does not utilise the extra power input it gets. For stall regulation, the blades cannot be turned, which then gives less energy extraction and the efficiency drops, when wind speeds are high, but protects the turbine from breaking and also does not need active control. For wind speeds over 20ms^(-1), the turbines shut down due to safety measures.  
This graph shows the effects of stalling and pitching through various wind speeds and is called the “wind turbine power curve”.

 
Electromagnetic induction:
In order to understand how the turbine generates electricity, electromagnetic induction is important to mention, since a generator is being used in this experiment. Inside the generator there is a magnetic field. An electric coil moves a magnetic field and through that a voltage is induced and an electric circuit is created. The faster the conductor or coil moves the bigger voltage induced. In a DC motor, the wires are changed via a commutator in order to get a rectified voltage.
To calculate the voltage is Faraday’s Law:
V=-N ΔBA/Δt

Here, V is voltage, N is the amount of turns in the coil, B is the external magnetic field, A is the area of the coil and together ΔBA is magnetic flux. Δt is time taken.
In wind turbines there is usually an electromagnet inside the generator, however in the bigger turbines a permanent magnet is secured, which creates its own magnetic field without the use of current from the turbine. A difference is that electromagnetism costs energy to maintain a field. 

To calculate efficiency:
Efficiency=output/input
For this experiment the output will be what the turbine produces onto the voltmeter and ammeter and is the product of voltage and current. The input is the power from the air coming in the area of the turbine. The efficiency us measured is power (W). 
Important to keep in mind is however, that of all the power in the area of the turbine blades, a maximum of 45\% of that power is actually used and average is 22-28\%. Since my turbine is far from an actual one, I would estimate the percentage to be rather low, so I am testing 10\%. 


\section{Method and Data Analysis}


\section{Uncertainties and Errors}


\section{Conclusion and Evaluation}


\begin{equation}
  gh
\label{WRITE A UNIQUE NAME HERE}
\end{equation} 
\subsection{Subheading}
\section{Conclusion}


%AGNES: DONT WRITE ANYTHING BELOW HERE

\nocite{*}
\newpage
\bibliography{references}
\bibliographystyle{unsrt}

%AGNES: PLEASE CHECK THIS STUFF OUT AT THE BOTTOM FOR AN EXPLANATION

%Anything that appears after a '%' is called a 'comment', that means that it will not appear in the PDF version of the document
%Below I have some code that you can copy and paste into your report when you need to:

%EQUATIONS
%This equation has all of the symbols you are likely to use in it, if you need a symbol that isn't written here, ask me or draw it in http://detexify.kirelabs.org/classify.html
%\begin{equation}
%1 + 2 \times 3 \div 4 = \frac{5}{2} = 2.5^{1} 
%\label{WRITE A UNIQUE NAME HERE}
%\end{equation}

%FIGURES
%This will put a figure in the text, the figure doesn't necessarily appear exactly where you will expect it to, but don't worry about that for now.
%Replace the Graph.pdf with your own figure name. Ideally the figures should all be PDF files, but some other image formats also work.
%The figures also have to be in the same folder as the .tex file.
%You can make the picture bigger or smaller by changing the number 0.7 - The maximum size this should be is 1, the smallest size is 0
%\begin{figure}
%  \centering
%  \includegraphics[width=0.7\textwidth]{img/Graph.pdf}
%  \caption{WRITE YOUR CAPTION FOR THE FIGURE HERE}
%  \label{WRITE A UNIQUE NAME HERE}
%\end{figure}

%REFERENCING EQUATIONS AND FIGURES
%If you want to refer to a particular equation or figure in the text, you should write:
%See Figure \ref{UNIQUE NAME HERE THAT CORRESPONDS TO THE NAME IN THE \label{...} OF THE EQUATION OR FIGURE}

%BIBLIOGRAPHY
%If you want to put a reference to the bibliography, you should write:
%...as explained by Albert Einstein \cite{UNIQUE NAME OF REFERENCE}.

%ADDING TO THE BIBLIOGRAPHY LIST
%This is done in the file references.bib, where a full example is given

\end{document}
